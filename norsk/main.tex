\documentclass[norsk]{beamer}				% frames
%\documentclass[notes, norsk]{beamer}		% frames + notes
%\documentclass[notes=only]{beamer}	% notes
\usepackage[utf8]{inputenc}		% for UTF8 characters
\usepackage{babel}		% manages typographical rules for English
\usepackage{csquotes}			% quotes that support babel
\usepackage{graphicx} 			% include graphics
\usepackage{verbatim} 			% include files with special characters
\usepackage{tabularx}			% fancy tables
\usepackage{booktabs}			% lines in tables
\usepackage{mathbbol}			% bold math symbols
\usepackage{float}				% fix object
\usepackage{color}				% colors
\usepackage{array} 				% for column alignment
\usepackage{listings}			% listings of code
\usepackage{hyperref}			% cross-reference
\usepackage{amsmath}			% various math symbols
\usepackage{physics}			% Dirac notation and matrices
\usepackage{amssymb}			% more math symbols
\usepackage{comment}			% in-line comments
\usepackage{makecell}			% new line in cell
\usepackage{subfig}				% sub figures
\usepackage{empheq}				% beautiful equation boxes
\usepackage{multirow}			% multirow in table
\usepackage{multicol}			% multicolumn in table
\usepackage{varwidth}			% rotate text in table
\usepackage{arydshln}   		% dashed lines in table
\usepackage{framed}
\usepackage{xargs} 
\usepackage{copyrightbox}		% copyright of images
\usepackage{mathpazo}  			% consistent nice math font 
\usepackage{mathtools}			% more math symbols
\usepackage{appendixnumberbeamer}	% appendix slides (backup slides)
\usepackage{pgfpages}			% separate note slides

% biblatex
\usepackage[backend=bibtex,
			%sorting=none,
			style=nature
]{biblatex}
\addbibresource{refs.bib}

% footnote position in slides
\usepackage[absolute,
			overlay
]{textpos}

% beamer presentation tool for Linux (display frames and notes separately)
%\usepackage[duration=35, 
%			lastminutes=10
%]{pdfpcnotes} 

% colors
\definecolor{maincolor}{RGB}{28, 39, 131}			% to be used in titles, progression wheel and references
\definecolor{empheqbackground}{RGB}{153, 153, 255}	% to be used as equation and quote background

\captionsetup[figure]{labelfont={color=maincolor}}
\captionsetup[table]{labelfont={color=maincolor}}

%empheq equation color box
\newlength\mytemplen
\newsavebox\mytempbox

\makeatletter
\newcommand\mybluebox{%
	\@ifnextchar[%]
	{\@mybluebox}%
	{\@mybluebox[0pt]}}

\def\@mybluebox[#1]{%
	\@ifnextchar[%]
	{\@@mybluebox[#1]}%
	{\@@mybluebox[#1][0pt]}}

\def\@@mybluebox[#1][#2]#3{
	\sbox\mytempbox{#3}%
	\mytemplen\ht\mytempbox
	\advance\mytemplen #1\relax
	\ht\mytempbox\mytemplen
	\mytemplen\dp\mytempbox
	\advance\mytemplen #2\relax
	\dp\mytempbox\mytemplen
	\colorbox{empheqbackground}{\hspace{1em}\usebox{\mytempbox}\hspace{1em}}}

% double quotes
\newcommand*\openquote{\makebox(40,-5){\scalebox{5}{``}}}
\newcommand*\closequote{\makebox(25,-22){\scalebox{5}{''}}}
\colorlet{shadecolor}{empheqbackground}
\makeatletter
\newif\if@right
\def\shadequote{\@righttrue\shadequote@i}
\def\shadequote@i{\begin{snugshade}\begin{quote}\openquote}
		\def\endshadequote{%
			\if@right\hfill\fi\closequote\end{quote}\end{snugshade}}
\@namedef{shadequote*}{\@rightfalse\shadequote@i}
\@namedef{endshadequote*}{\endshadequote}
\makeatother

\input{frames.tex}
\input{defs.tex}

%%%%%%%%%%%%%%%%%%%%%%%%%%%%%%%%%%%%%%%%%%%%%%%%%%%%%%%%%%%%%%%%%%%%%%%%%%%%%%%%%%%%%%%%%%%%%%%%%%%%%%%%%%%%%%%%%%%%%
% SPECIFY INFORMATION TO BE USED IN FRONT FRAME
%%%%%%%%%%%%%%%%%%%%%%%%%%%%%%%%%%%%%%%%%%%%%%%%%%%%%%%%%%%%%%%%%%%%%%%%%%%%%%%%%%%%%%%%%%%%%%%%%%%%%%%%%%%%%%%%%%%%%

\newcommand{\mtitle}{Eksempel: Kvantemekanikk}
\newcommand{\mauthor}{Even Marius Nordhagen}
\newcommand{\mmail}{evenmn@fys.uio.no}
\newcommand{\massignn}{.}

\begin{document}

\frontframe

\note{
	\begin{itemize}
		\item This is an example presentation about quantum mechanics
		\item The front frame is generated using \textit{frontframe}
		\item Note also that the notes can be turned on and off in the first line of this file
	\end{itemize}
}

\mframe{Oversikt}{}{
	\begin{itemize}
		\setlength\itemsep{1em}	% This line specifies the spacing between bullet points
		\item Schrödinger's ligning
		\item Sannsynlighetsfordelingen
		\item <2> Referanser
	\end{itemize}
		\note<1>{Her er en oversikt over presentasjonen}
		\note<2>{Notis for siste kulepunkt}
}

\mframe{}{}{
	\begin{shadequote}{
			Den generelle kvanteteorien er nesten komplett... ...De underliggende fysiske lovene nødvendig for den matematiske teorien for en stor del av fysikken og hele kjemien er kjent, og utfordringen ligger i at anvendelsen av disse lovene fører til ligninger som er alt for kompliserte til at de kan løses. \par Paul M. Dirac, \emph{Quantum Mechanics of Many-electron Systems} \supercite{dirac_paul_adrien_maurice_quantum_1929}}
	\end{shadequote}
}

\note{Lysbilde uten tittel eller undertittel}

\mframe{Schrödinger's ligning}{Den tidsuavhengige Schrödingerligningen}{
	Den tidsuavhengige Schrödingerligningen er gitt ved
	\begin{empheq}[box={\mybluebox[5pt]}]{equation}
		\hat{\mathcal{H}}\Psi_n=\varepsilon_n\Psi_n
	\end{empheq}
	hvor $\hat{\mathcal{H}}$ er Hamiltonoperatoren, $\Psi_n$ er bølgefunksjonen og $\varepsilon_n$ er den tilknyttede energien \supercite{schrodinger_undulatory_1926}. 
}
\note{Her er et enkelt lysbilde med undertittel}

\mframe{Sannsynlighetsfordelingen}{}{
	Sannsynlighetsfordelingen i kvantemekanikk er gitt ved
	\begin{empheq}[box={\mybluebox[5pt]}]{equation}
		P(\boldsymbol{r})=\frac{\Psi_n(\boldsymbol{r})^*\Psi_n(\boldsymbol{r})}{\int d\boldsymbol{r}\Psi_n(\boldsymbol{r})^*\Psi_n(\boldsymbol{r})}
	\end{empheq}
	hvor $\boldsymbol{r}$ er et sett av romlige- og spinkoordinater \supercite{born_zur_1926}.
	\pause
	Ofte antar vi at bølgefunksjonen er normalisert, noe som forenkler ligningen:
	\begin{equation}
	P(\boldsymbol{r})=\Psi_n(\boldsymbol{r})^*\Psi_n(\boldsymbol{r}).
	\end{equation}
	\note<1>{Dette er en enkelt lysbilde uten undertittel}
	\note<2>{'pause'-funksjonen kan brukes til å legge elementer til lysbildet}
}

\titleframe{Tusen takk!}

\note{'titleframe' inneholder kun en sentrert tekst (må ikke forveksles med 'frontframe')}

\mframe{References}{}{
	\printbibliography
}

%\appendix

% Backup slides

%\titleframe{Some Backup Slides}

%\mframe{Backup slides}{}{
%	\begin{figure}
%	\centering
%	\input{../tikz/biasvariancedecomp.tex}
%	\end{figure}
%}


\end{document}
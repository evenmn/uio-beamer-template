\usepackage[utf8]{inputenc}		% for UTF8 characters
\usepackage[english]{babel}		% manages typographical rules for English
\usepackage{csquotes}			% quotes that support babel
\usepackage{graphicx} 			% include graphics
\usepackage{verbatim} 			% include files with special characters
\usepackage{tabularx}			% fancy tables
\usepackage{booktabs}			% lines in tables
\usepackage{mathbbol}			% bold math symbols
\usepackage{float}				% fix object
\usepackage{color}				% colors
\usepackage{array} 				% for column alignment
\usepackage{listings}			% listings of code
\usepackage{hyperref}			% cross-reference
\usepackage{amsmath}			% various math symbols
\usepackage{physics}			% Dirac notation and matrices
\usepackage{amssymb}			% more math symbols
\usepackage{comment}			% in-line comments
\usepackage{makecell}			% new line in cell
\usepackage{subfig}				% sub figures
\usepackage{empheq}				% beautiful equation boxes
\usepackage{multirow}			% multirow in table
\usepackage{multicol}			% multicolumn in table
\usepackage{varwidth}			% rotate text in table
\usepackage{arydshln}   		% dashed lines in table
\usepackage{framed}
\usepackage{xargs} 
\usepackage{copyrightbox}		% copyright of images
\usepackage{mathpazo}  			% consistent nice math font 
\usepackage{mathtools}			% more math symbols
\usepackage{appendixnumberbeamer}	% appendix slides (backup slides)
\usepackage{pgfpages}			% separate note slides

% biblatex
\usepackage[backend=bibtex,
			%sorting=none,
			style=nature
]{biblatex}
\addbibresource{refs.bib}

% footnote position in slides
\usepackage[absolute,
			overlay
]{textpos}

% beamer presentation tool for Linux (display frames and notes separately)
\usepackage[duration=35, 
			lastminutes=10
]{pdfpcnotes} 

% colors
\definecolor{maincolor}{RGB}{28, 39, 131}			% to be used in titles, progression wheel and references
\definecolor{empheqbackground}{RGB}{153, 153, 255}	% to be used as equation and quote background

\captionsetup[figure]{labelfont={color=maincolor}}
\captionsetup[table]{labelfont={color=maincolor}}

%empheq equation color box
\newlength\mytemplen
\newsavebox\mytempbox

\makeatletter
\newcommand\mybluebox{%
	\@ifnextchar[%]
	{\@mybluebox}%
	{\@mybluebox[0pt]}}

\def\@mybluebox[#1]{%
	\@ifnextchar[%]
	{\@@mybluebox[#1]}%
	{\@@mybluebox[#1][0pt]}}

\def\@@mybluebox[#1][#2]#3{
	\sbox\mytempbox{#3}%
	\mytemplen\ht\mytempbox
	\advance\mytemplen #1\relax
	\ht\mytempbox\mytemplen
	\mytemplen\dp\mytempbox
	\advance\mytemplen #2\relax
	\dp\mytempbox\mytemplen
	\colorbox{empheqbackground}{\hspace{1em}\usebox{\mytempbox}\hspace{1em}}}

% double quotes
\newcommand*\openquote{\makebox(40,-5){\scalebox{5}{``}}}
\newcommand*\closequote{\makebox(25,-22){\scalebox{5}{''}}}
\colorlet{shadecolor}{empheqbackground}
\makeatletter
\newif\if@right
\def\shadequote{\@righttrue\shadequote@i}
\def\shadequote@i{\begin{snugshade}\begin{quote}\openquote}
		\def\endshadequote{%
			\if@right\hfill\fi\closequote\end{quote}\end{snugshade}}
\@namedef{shadequote*}{\@rightfalse\shadequote@i}
\@namedef{endshadequote*}{\endshadequote}
\makeatother
